\documentclass{article}
\usepackage[left=30mm, top=15mm, right=30mm, bottom=25mm, nohead, nofoot]{geometry}
\usepackage[T2A]{fontenc}
\usepackage[utf8]{inputenc}
\usepackage[russian]{babel}
\usepackage{amsfonts}
\usepackage{amsmath}

\title{Numerical Python -- ODE -- Exercise 4}
\date{}
\author{}
 
\begin{document}

\maketitle

Имуются точное и приближенное решения $I\cos(\omega t)$ и $I\cos(\tilde{\omega}t)$. В качестве погрешности фазы будем понимать задержку по времени пика I точного решения и соответствующего пика приближенного решения после m периодов колебаний. Конкретнее, будем рассматривать локальные максимумы.

\section{Аналитическое доказательство}
Пики для аналитического и численного решения задаются формулами
$$t_m=\cfrac{2\pi}{\omega}m$$
и
$$\tilde{t}_m=\cfrac{2\pi}{\tilde{\omega}}m.$$

Погрешность фазы получается равной
$$\varepsilon_m = t_m - \tilde{t}_m=(\cfrac{2\pi}{\omega} - \cfrac{2\pi}{\tilde{\omega}})m,$$
то есть является линейной функцией относительно $m$.

\section{Численное доказательство}
Предположим, что погрешность фазы $\varepsilon_m$ после $m$ периодов колебаний представима в виде
\begin{equation}\label{eq:assumption}
\varepsilon_m = k m.
\end{equation}
Тогда, в силу того, что численное решение представляется сеточной функцией, представление (\ref{eq:assumption}) верно с точностью до $\tau$, где $\tau$ -- величина шага. То есть
\begin{equation}
k m - \tau \leq \varepsilon_m \leq k m + \tau.
\end{equation}

Проведём серию вычислений $\cfrac{e_m}{m}$ для различных $m$. Если предположение верно, то
$$\cfrac{e_m}{m} \approx k$$
с точностью до $\cfrac{\tau}{m}$.


\end{document}
